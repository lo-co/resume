\documentclass[margin,line]{res}


\oddsidemargin -.5in
\evensidemargin -.5in
\textwidth=6.0in
\itemsep=0in
\parsep=0in
% if using pdflatex:
%\setlength{\pdfpagewidth}{\paperwidth}
%\setlength{\pdfpageheight}{\paperheight} 

\newenvironment{list1}{
  \begin{list}{\ding{113}}{%
      \setlength{\itemsep}{0in}
      \setlength{\parsep}{0in} \setlength{\parskip}{0in}
      \setlength{\topsep}{0in} \setlength{\partopsep}{0in} 
      \setlength{\leftmargin}{0.17in}}}{\end{list}}
\newenvironment{list2}{
  \begin{list}{$\bullet$}{%
      \setlength{\itemsep}{0in}
      \setlength{\parsep}{0in} \setlength{\parskip}{0in}
      \setlength{\topsep}{0in} \setlength{\partopsep}{0in} 
      \setlength{\leftmargin}{0.2in}}}{\end{list}}


\begin{document}

\name{Mathews S Richardson \vspace*{.1in}}

\begin{resume}
\section{\sc Contact Information}
\vspace{.05in}
\begin{tabular}{@{}p{2in}p{4in}}
2313 Rockspray Ct             & {\it Phone:}  (303)931-6760 \\            
Longmont, CO 80503  & {\it E-mail:}    matt.richardson@msrconsults.com \\      
\end{tabular}

\section{\sc Education}
{\bf Colorado State University}, Fort Collins, CO\\
%{\em Department of Statistics} 
\vspace*{-.1in}
\begin{list1}
\item[] Ph.D., Atmospheric Science, May 2009
\vspace*{.05in}
\item[] Dissertation Title:  ``Making Real Time Measurements of Ice Nuclei Concentrations at Upper Tropospheric Temperatures: Extending the Capabilities of the Continuous Flow Diffusion Chamber'' 
%\item Dissertation Topic:  ``Hierarchical Models for Multiple Ratings
%  in Performance-Based\\ \hspace*{1.23in} Student Assessments.'' 
\item[] Advisor:  Sonia Kreidenweis
\end{list1}

{\bf Texas A\&M University}, College Station, TX\\
%{\em Department of Mathematics and Statistics} 
\vspace*{-.1in}
\begin{list1}
\item[] M.S., Mechanical Engineering,  August, 2003
\vspace*{0.05in}
\item[] Thesis Title: ``A System for Continuous Sampling of Bioaerosols Generated by a Postal Sorting Machine''
\item[] Advisor: Andrew McFarland
\end{list1}

{\bf University of Texas}, Austin, TX\\
%{\em Department of Mathematics and Statistics} 
\vspace*{-.1in}
\begin{list1}
\item[] B.S., Mechanical Engineering,  May, 1999
\end{list1}

\section{\sc Certifications}
\begin{list2}
\item Certified LabVIEW Developer
\end{list2}

\section{\sc Computer Skills} 
\begin{list2}
\item Languages:  C++, C\#, Ada, Python, Java, Javascript, HTML, CSS, PHP, elm
\item Applications: MATLAB, \LaTeX, common Windows
  database, spreadsheet, and presentation software, FLuent, SolidWorks, LabVIEW, Visual Studio, WebStorm, PyCharm, IntelliJ, NetBeans
\item Operating Systems:  Unix/Linux, Windows.\\ 
\end{list2}

\section{\sc Experience}
{\bf MSR Consulting, LLC}, Longmont, CO

\vspace{-.3cm}
{\em Software Consultant and Systems Integrator} \hfill {\bf January 2013 - present}\\
Owner and primary consultant of this National Instruments Alliance Partner company.  Work with a variety of companies to develop data acquisition and analysis solutions at all project phases, from product selection to architecting of the system to execution of the software.  Past and present customers include the UK Met Office, Altitude Control Technologies, Independent Testing Laboratories and Handix Scientific.  


{\bf Cooperative Institute for Research in Environmental Sciences}, Boulder, CO

\vspace{-.3cm}
%\vspace{-.1cm}
{\em Research Associate} \hfill {\bf December 2008 - present}\\
Serve as principle software engineer in the Cloud and Aerosol Processes group in the Chemical Sciences Division of the National Oceanic and Atmospheric Administration in Boulder.  
\vspace*{.05in} 
\begin{list2}
	\item Provide recommendations and assemble off the shelf systems for instrument data acquisition systems.
	\item Architect and develop applications for acquiring data from high-performance, one-of-a-kind instrumentation for sampling and analyzing atmospheric aerosol using a variety of languages.
	\item Support airborne and ground-based campaigns utilizing instrumentation developed in the laboratory.
	\item Analyze data and synthesize results from experiments for presentation and publication.
\end{list2}


{\bf Colorado State University}, Fort Collins, CO

\vspace{-.3cm}
%\vspace{-.1cm}
{\em Research Assistant} \hfill {\bf August 2003 - December 2008}\\
Conducted research in the atmospheric chemistry group under Sonia Kreidenweis concerning cold cloud formation. 
\vspace*{.05in}  
\begin{list2}
	\item Operated and supported further development of the Continuous Flow Diffusion Chamber (CFDC).
	\item Participated in several ground- and air-based campaigns utilizing the CFDC.
	\item Developed fluid dynamic models for studying heat and mass transport in the CFDC in connection with a dissertation.
	\item Conducted ground based experiments to validate model findings of CFDC performance.
	\item Presented results from experiments and campaigns in a final dissertation.
\end{list2}


{\bf Texas A\&M University}, College Station, TX

\vspace{-.3cm}
%\vspace{-.1cm}
{\em Research Assistant} \hfill {\bf Janary 2002 - August 2003}\\
Conducted research in the Aerosol Technology Laboratory under Andrew McFarland concerning the sampling of bioaerosols.
\vspace*{.05in} 
\begin{list2}
\item Worked with engineers from Siemens to develop a system for sampling bioaerosol from a mail sorting machine.
\item Collected and analyzed samples using biological techniques including culturing bacteria.
\item Assimilated analyzed data for presentation to the United States Post Office as well as a thesis.
\end{list2} 


{\bf Texas A\&M University}, College Station, TX

\vspace{-.3cm}
%\vspace{-.1cm}
{\em Graduate Teaching Assistant} \hfill {\bf August 2001- December 2001}\\
Served as graduate assistant for engineering thermodynamics (MEEN 315):
\vspace*{.05in} 
\begin{list2}
\item Graded homework and exams.
\item Proctored exams.
\item Conducted study sessions.
\item Taught when professor was unavailable.
\end{list2}


{\bf Geologics}, Houston, TX

\vspace{-.3cm}
%\vspace{-.1cm}
{\em Software Consultant} \hfill {\bf July 2001}\\
Temporary consulting position to support training operations associated with software developed in capacity with Booz, Allen and Hamilton.


{\bf Booz, Allen, and Hamilton}, Houston, TX

\vspace{-.3cm}
%\vspace{-.1cm}
{\em Software Engineer} \hfill {\bf December 1999 - May 2001}\\
Developed training models using Ada for the International Space Station in the Structural and Mechanical group as a subcontractor to Raytheon.  Assisted in operation of large scale simulations.

%------------------------------------------------

\section{\sc Publications}
{\bf Richardson, M. S.}, P. J. DeMott, S. M. Kreidenweis, D. Cziczo, E. J. Dunlea, J. L. Jimenez, D. Thomson, L. Ashbaugh, R. D. Borys, D. Westphal, G. Casuccio and T. Lersch (2007), Measurements of heterogeneous ice nuclei in the western United States in springtime and their relation to aerosol characteristics, J. Geophys. Res., 112, D02209, doi:10.1029/2006JD007500.

Prenni, A. J., P. J. DeMott, C. Twohy, M. R Poellot, S. M. Kreidenweis, D. C. Rogers, S. D. Brooks, {\bf M. S. Richardson}, and A. J. Heymsfield (2007), Examinations of ice formation processes in Florida cumuli using ice nuclei measurements of anvil ice crystal particle residues, J. Geophys. Res., 112, D10221, doi:10.1029/2006JD007549.

{\bf Richardson, M. S.}, DeMott, P. J., Kreidenweis, S. M., Petters, M. D., and Carrico, C. M. (2010), Observations of ice nucleation by ambient aerosol in the homogeneous freezing regime, Geophys. Res. Lett., 37, L04806.

DeMott, P. J., Prenni, A. J., Liu, X., Kreidenweis, S. M., Petters, M. D., Twohy, C. H., {\bf Richardson, M. S.}, Eidhammer, T., and Rogers, D. C. (2010). Predicting global atmospheric ice nuclei distributions and their impacts on climate, Proc. Natl. Acad. Sci., 107 (25), 11217-11222.
Paciorek, C.J., J.S. Risbey, V. Ventura, and R.D.Rosen. 2002. Multiple indices of Northern Hemisphere Cyclone
Activity, Winters 1949-1999, J. of Climate, 15:1573-1590.

DeMott, P.J., A.J. Prenni, X. Liu, S.M. Kreidenweis, M.D. Petters, C.H. Twohy, {\bf M.S. Richardson}, T. Eidhammer and D.C. Rogers (2010), Predicting global atmospheric ice nuclei distributions and their impacts on climate. Proc. Natl. Acad. Sci. U. S. A., 107 (25) 11217-11222, issn: 0027-8424, ids: 614KS, doi:10.1073/pnas.0910818107.

Langridge, J.M., {\bf M.S. Richardson}, D. Lack, D. Law and D.M. Murphy (2011), Aircraft Instrument for Comprehensive Characterization of Aerosol Optical Properties, Part I: Wavelength-Dependent Optical Extinction and Its Relative Humidity Dependence Measured Using Cavity Ringdown Spectroscopy. Aerosol Sci. Technol., 45 (11) 1305-1318, issn: 0278-6826, ids: 816MF, doi:10.1080/02786826.2011.592745.

Lack, D.A., {\bf M.S. Richardson}, D. Law, J.M. Langridge, C.D. Cappa, R.J. McLaughlin and D.M. Murphy (2012), Aircraft Instrument for Comprehensive Characterization of Aerosol Optical Properties, Part 2: Black and Brown Carbon Absorption and Absorption Enhancement Measured with Photo Acoustic Spectroscopy. Aerosol Sci. Technol., 46 (5) 555-568, issn: 0278-6826, ids: 912CQ, doi:10.1080/02786826.2011.645955.

Langridge, J.M., {\bf M.S. Richardson}, D.A. Lack, C.A. Brock and D.M. Murphy (2013), Limitations of the Photoacoustic Technique for Aerosol Absorption Measurement at High Relative Humidity. Aerosol Sci. Technol., 47 (11) 1163-1173, issn: 0278-6826, ids: 218VD, doi:10.1080/02786826.2013.827324.

Wagner, N.L., C.A. Brock, W.M. Angevine, A. Beyersdorf, P. Campuzano-Jost, D. Day, J.A. de Gouw, G.S. Diskin, T.D. Gordon, M.G. Graus, J.S. Holloway, G. Huey, J.L. Jimenez, D.A. Lack, J. Liao, X. Liu, M.Z. Markovic, A.M. Middlebrook, T. Mikoviny, J. Peischl, A.E. Perring, {\bf M.S. Richardson}, T.B. Ryerson, J.P. Schwarz, C. Warneke, A. Welti, A. Wisthaler, L.D. Ziemba and D.M. Murphy (2015), In situ vertical profiles of aerosol extinction, mass, and composition over the southeast United States during SENEX and SEAC(4)RS: observations of a modest aerosol enhancement aloft. Atmos. Chem. Phys., 15 (12) 7085-7102, issn: 1680-7316, ids: CL6ZC, doi:10.5194/acp-15-7085-2015.

Brock, C.A., N.L. Wagner, B.E. Anderson, A.R. Attwood, A. Beyersdorf, P. Campuzano-Jost, A.G. Carlton, D.A. Day, G.S. Diskin, T.D. Gordon, J.L. Jimenez, D.A. Lack, J. Liao, M.Z. Markovic, A.M. Middlebrook, N.L. Ng, A.E. Perring, {\bf M.S. Richardson}, J.P. Schwarz, R.A. Washenfelder, A. Welti, L. Xu, L.D. Ziemba and D.M. Murphy (2016), Aerosol optical properties in the southeastern United States in summer - Part 1: Hygroscopic growth. Atmos. Chem. Phys., 16 (8) 4987-5007, issn: 1680-7316, ids: DN3BK, doi:10.5194/acp-16-4987-2016.

Gordon, T.D., N.L. Wagner, {\bf M.S. Richardson}, D.C. Law, D. Wolfe, E.W. Eloranta, C.A. Brock, F. Erdesz and D.M. Murphy (2015), Design of a Novel Open-Path Aerosol Extinction Cavity Ringdown Spectrometer. Aerosol Sci. Technol., 49 (9) 716-725, issn: 0278-6826, ids: CO4VX, doi:10.1080/02786826.2015.1066753.

Warneke, C., M. Trainer, J.A. de Gouw, D.D. Parrish, D.W. Fahey, A.R. Ravishankara, A.M. Middlebrook, C.A. Brock, J.M. Roberts, S.S. Brown, J.A. Neuman, B.M. Lerner, D. Lack, D. Law, G. Hubler, I. Pollack, S. Sjostedt, T.B. Ryerson, J.B. Gilman, J. Liao, J. Holloway, J. Peischl, J.B. Nowak, K.C. Aikin, K.E. Min, R.A. Washenfelder, M.G. Graus, {\bf M. Richardson}, M.Z. Markovic, N.L. Wagner, A. Welti, P.R. Veres, P. Edwards, J.P. Schwarz, T. Gordon, W.P. Dube, S.A. McKeen, J. Brioude, R. Ahmadov, A. Bougiatioti, J.J. Lin, A. Nenes, G.M. Wolfe, T.F. Hanisco, B.H. Lee, F.D. Lopez-Hilfiker, J.A. Thornton, F.N. Keutsch, J. Kaiser, J.Q. Mao and C.D. Hatch (2016), Instrumentation and measurement strategy for the NOAA SENEX aircraft campaign as part of the Southeast Atmosphere Study 2013. Atmos. Meas. Tech., 9 (7) 3063-3093, issn: 1867-1381, ids: DS9HQ, doi: 10.5194/amt-9-3063-2016.
\end{resume}
\end{document}




